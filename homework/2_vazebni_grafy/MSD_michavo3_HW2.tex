\documentclass[twoside]{article}
\usepackage[a4paper]{geometry}
\geometry{verbose,tmargin=2.5cm,bmargin=2cm,lmargin=2cm,rmargin=2cm}
\usepackage{fancyhdr}
\pagestyle{fancy}

% nastavení pisma a~češtiny
\usepackage{lmodern}
\usepackage[T1]{fontenc}
\usepackage[utf8]{inputenc}
\usepackage[czech]{babel}

% odkazy
\usepackage{url}

\usepackage{float}
% vícesloupcové tabulky
\usepackage{multirow}
\usepackage{amssymb}
\usepackage{gensymb}
\usepackage{bbold}
\usepackage{mathtools}
\usepackage{commath}
\usepackage{siunitx}

% vnořené popisky obrázků
\usepackage{subcaption}

% automatická konverze EPS 
\usepackage{graphicx} 
\usepackage{epstopdf}
\usepackage{amsmath}
\epstopdfsetup{update}

% odkazy a~záložky
\usepackage[unicode=true, bookmarks=true,bookmarksnumbered=true,
bookmarksopen=false, breaklinks=false,pdfborder={0 0 0},
pdfpagemode=UseNone,backref=false,colorlinks=true] {hyperref}

% Poznámky při překladu
\usepackage{xkeyval}	% Inline todonotes
\usepackage[textsize = footnotesize]{todonotes}
\presetkeys{todonotes}{inline}{}
\graphicspath{{./images}}

%https://tex.stackexchange.com/questions/2783/bold-calligraphic-typeface
\DeclareMathAlphabet\mathbfcal{OMS}{cmsy}{b}{n}

% Zacni sekci slovem ukol
\renewcommand{\thesection}{Úkol \arabic{section}}
% enumerate zacina s pismenem
\renewcommand{\theenumi}{\alph{enumi}}

% smaz aktualni page layout
\fancyhf{}
% zahlavi
\usepackage{titling}
\fancyhf[HC]{\thetitle}
\fancyhf[HLE,HRO]{\theauthor}
\fancyhf[HRE,HLO]{\today}
 %zapati
\fancyhf[FLE,FRO]{\thepage}

\title{Modelování a simulace dynamických systémů - úkol č. 2}
\author{Vojtěch Michal}

\begin{document}
\maketitle

\section{Doplňte poloviční šipku do vazebního grafu}
%Doplňte správně poloviční šipku do daného vazebního grafu pro jednoduchý elektronický či mechanický systém
Uvažujte situaci, kdy za provaz zvedáte do výšky nějaké závaží.
Do náčrtku této situace zakreslete Vaši volbu kladného směru pohybu (rychlosti) a v souhlasu
s touto volbou dokreslete do vazebního grafu JÁ \tikz \draw (0,0) -- (1,0); ZÁVAŽÍ poloviční šipku.

\vspace{3cm}

\section{Spočtěte modul transformátoru}
%Spočtěte modul transformátoru pro zadaný systém z elektrické, mechanické či hydraulické domény
Modelujte vícestupňovou převodovku pomocí kaskády transformátorů. Najděte modul transformátorů (podle konvence) a modul jednoho ekvivalentního transformátoru. Čísla v obrázku značí počet zubů.

\begin{figure}[ht]
	\centering
	\includegraphics[width=0.5\textwidth]{./figures/prevody.pdf}
\end{figure}

\textbf{Řešení:} Dvojice ozubených kol sdílející společnou hřídel sdílí i úhlovou rychlost otáčení $\omega$. Dvojice kol dotýkající se zuby vytváří zobecnělý transformátor.
Pro každé kolo je počet zubů, které projdou bodem dotyku za sekundu, roven $f = N \frac{\omega}{2 \pi}$, kde $\omega$ je úhlová rychlost otáčení daného kola a $N$ je počet jeho zubů.
Protože se v bodě dotyku zuby obou kol střídají, frekvence $f$ je zachována. Odtud pro kola 1 a 2 platí
\begin{equation}
	\begin{split}
		f_1 &= f_2 \\
		N_1 \frac{\omega_1}{2 \pi} &= N_2 \frac{\omega_2}{2 \pi} \\
		\omega_2 &= \underbrace{\frac{N_1}{N_2}}_{T} \omega_1,
	\end{split}
\end{equation}
analogicky poté i pro ostatní páry. Ve směru od vstupu k výstupu
proto mají transformátory moduly $T_1 = \frac{24}{12} = 2$, $T_2 = \frac{28}{15} = $ a $T_3 = \frac{40}{20} = 2$.
Protože převod ozubenými koly současně mění směr rotace, bylo by možné uvažovat záporné moduly $T_i$.
Přirozenější se však jeví uvažovat kladný směr $\omega_2$ v opačném směru než kladný směr rotace $\omega_1$,
tím zůstane modul dílčích transofrmátorů kladný.

Po spojení vytvoří tři transformátory jeden ekvivalentní nový s modulem $T = T_1 T_2 T_3 = 7.4\bar{6}$.

\section{Určete modul gyrátoru modelujícího stejnosměrný motor}
%Použijte gyrátor pro modelování běžného kartáčového stejnosměrného motoru s permanentním magnetem. Pro zadané fyzikální parametry motoru určete modul gyrátoru
Ke stejnosměrnému motoru je připojena zátěž, která je popsána rovnicí $M=5\cdot10^{-5}\omega$ ($\omega$ je úhlová rychlost a $M$ brzdný moment) .
V ustáleném stavu teče do motoru proud 300\,mA a má otáčky 5700\,rpm. Odpor vinutí je 20\,$\Omega$ a indukčnost 200\,$\mu$H. Určete modul gyrátoru, který tento motor modeluje. 

\vspace{6cm}

\section{Maximální přenos výkonu ze zdroje do spotřebiče}
%Pro zadaný zdroj a spotřebič s jejich výkonovými (zatěžovacími) charakteristikami použijte zobecněný transformátor pro zajištění maximálního přenosu výkonu ze zdroje do spotřebiče
Hydraulická pumpa dodávající konstantní tlak 17\,MPa s maximálním průtokem 0,5\,m$^3$/s je připojena k hydraulickému pístu. Píst pohybuje se zátěží, která je charakterizována lineárním vztahem mezi rychlostí a silou a to takovým, že při působení síly 18\,kN se pohybuje rychlostí 30\,mm/s. Jakým prvkem je píst ve vazebním grafu modelován, jaký má modul pro maximální rychlost pohybu a maximální přenos výkonu a čemu tento modul fyzicky odpovídá?

\begin{figure}[h!]
	\centering
	\includegraphics[width=0.65\textwidth]{./figures/charakteristiky.pdf}
	\caption{Charakteristika pumpy a zátěže.}
\end{figure}


\vspace{6cm}

\section{Odvoďte vztah pro energii akumulovanou v prvku typu zobecněná setrvačnost I}
A to jak pro nelineární tak i pro lineární prvek.

\textbf{Řešení:} 
Pro zobecněnou setrvačnost z definice platí 
\begin{equation}
	\dot{q} = \dot{q}(p),
	\label{eq:setrvacnost_def}
\end{equation}
kde $\dot{q}$ je \textit{zobecnělý tok} a $p = \int e \text{d}t$ je \textit{zobecnělá hybnost}.
Akumulovaná (kinetická) energie je rovna práci vykonané během "rozpohybovávání" prvku
\begin{equation}
	\mathcal{T} = \int_{t_1}^{t_2} \underbrace{e(t) \dot{q}(t)}_{P(t)} \text{d}t,
	\label{eq:setrvacnost_obecne}
\end{equation}
s využitím vztahu \eqref{eq:setrvacnost_def} obecně platí
\begin{equation}
	\mathcal{T} = \int_{t_1}^{t_2} \frac{\text{d}p(t)}{\text{d}t} \dot{q}(p) \text{d}t = \int_{p_1}^{p_2} \dot{q}(p) \text{d}p.
	\label{eq:setrvacnost_energie_obecne}
\end{equation}
V lineárním případě má definiční vztah \eqref{eq:setrvacnost_def} tvar $\dot{q} = \frac{1}{I} p$,
poté je možné obecný vztah pro energii \eqref{eq:setrvacnost_energie_obecne} upravit dále na
\begin{equation}
	\mathcal{T} = \frac{1}{I} \int_{p_1}^{p_2} p \text{d}p = \frac{1}{2I} (p_2^2 - p_1^2).
\end{equation}
Za předpokladu nulových počátečních podmínek ($p_1 = 0$) získáme známý vztah
\begin{equation}
	\mathcal{T} = \frac{1}{2I}p^2 = \frac{1}{2} I \dot{q}^2.
\end{equation}


\section{Odvoďte vztah pro energii akumulovanou v prvku typu zobecněná poddajnost C}
A to jak pro nelineární tak i pro lineární prvek.

\textbf{Řešení:} 
Pro zobecněnou poddajnost z definice platí 
\begin{equation}
	e = e(q),
	\label{eq:poddajnost_def}
\end{equation}
kde $e$ je \text{zobecnělé úsilí} a $q = \int \dot{q}\,\text{d}t$ je \textit{zobecnělá výchylka}.
Akumulovaná (potenciální) energie je rovna práci vykonané během "nabíjení" prvku
\begin{equation}
	\mathcal{V} = \int_{t_1}^{t_2} \underbrace{e(t) \dot{q}(t)}_{P(t)} \text{d}t,
	\label{eq:poddajnosT_obecne}
\end{equation}
s využitím vztahu \eqref{eq:poddajnost_def} obecně platí
\begin{equation}
	\mathcal{V} = \int_{t_1}^{t_2} e(q) \frac{\text{d}q}{\text{d}t} \text{d}t = \int_{q_1}^{q_2} e(q) \text{d}q.
	\label{eq:poddajnost_energie_obecne}
\end{equation}
V lineárním případě má definiční vztah \eqref{eq:poddajnost_def} tvar $e = \frac{1}{C} q$,
poté je možné obecný vztah pro energii \eqref{eq:poddajnost_energie_obecne} upravit dále na
\begin{equation}
	\mathcal{V} = \frac{1}{C} \int_{q_1}^{q_2} q \text{d}q = \frac{1}{2C} (q_2^2 - q_1^2).
\end{equation}
Za předpokladu nulových počátečních podmínek ($q_1 = 0$) získáme známý vztah
\begin{equation}
	\mathcal{V} = \frac{1}{2C}q^2 = \frac{1}{2} C e^2.
\end{equation}

\end{document}
