\documentclass[twoside]{article}
\usepackage[a4paper]{geometry}
\geometry{verbose,tmargin=2.5cm,bmargin=2cm,lmargin=2cm,rmargin=2cm}
\usepackage{fancyhdr}
\pagestyle{fancy}

% nastavení pisma a~češtiny
\usepackage{lmodern}
\usepackage[T1]{fontenc}
\usepackage[utf8]{inputenc}
\usepackage[czech]{babel}

% odkazy
\usepackage{url}

\usepackage{float}
% vícesloupcové tabulky
\usepackage{multirow}
\usepackage{amssymb}
\usepackage{gensymb}
\usepackage{bbold}
\usepackage{mathtools}
\usepackage{commath}

% vnořené popisky obrázků
\usepackage{subcaption}

% automatická konverze EPS 
\usepackage{graphicx} 
\usepackage{epstopdf}
\usepackage{amsmath}
\epstopdfsetup{update}

% odkazy a~záložky
\usepackage[unicode=true, bookmarks=true,bookmarksnumbered=true,
bookmarksopen=false, breaklinks=false,pdfborder={0 0 0},
pdfpagemode=UseNone,backref=false,colorlinks=true] {hyperref}

% Poznámky při překladu
\usepackage{xkeyval}	% Inline todonotes
\usepackage[textsize = footnotesize]{todonotes}
\presetkeys{todonotes}{inline}{}
\graphicspath{{./images}}

%https://tex.stackexchange.com/questions/2783/bold-calligraphic-typeface
\DeclareMathAlphabet\mathbfcal{OMS}{cmsy}{b}{n}

% Zacni sekci slovem ukol
\renewcommand{\thesection}{Úkol \arabic{section}}
% enumerate zacina s pismenem
\renewcommand{\theenumi}{\alph{enumi}}

% smaz aktualni page layout
\fancyhf{}
% zahlavi
\usepackage{titling}
\fancyhf[HC]{\thetitle}
\fancyhf[HLE,HRO]{\theauthor}
\fancyhf[HRE,HLO]{\today}
 %zapati
\fancyhf[FLE,FRO]{\thepage}

% údaje o autorovi
\title{Modelování a simulace dynamických systémů - úkol č. 9}
\author{Vojtěch Michal}
\date{\today}

\begin{document}

\maketitle

Zadání je dostupné na \url{https://moodle.fel.cvut.cz/mod/assign/view.php?id=200293}. \\

Z předchozího řešení je známo

\begin{equation}
	R_1^0 = \begin{bmatrix}
		\cos(\theta_1) & 0 & -\sin(\theta_1) \\
		\sin(\theta_1) & 0 & \cos(\theta_1) \\
		0 & -1 & 0 
	\end{bmatrix},
\end{equation}

\begin{equation}
	R_2^1 = \begin{bmatrix}
		\cos(\theta_2) & -\sin(\theta_2) & 0 \\
		\sin(\theta_2) & \cos(\theta_2) & 0 \\
		0 & 0 & 1 
	\end{bmatrix}.
\end{equation}

Máme dánu délky ramen $l_1$, $l_2$, a hmotnost ramen $m_1$, $m_2$.
Jsou dány matice momentů setrvačnosti ramen v příslušných souřadných soustavách
\begin{equation}
	I_1^1 = \begin{bmatrix}
		I_{x_1} & 0 &0 \\
		0 & I_{y_1} &0 \\
		0 & 0 & I_{z_1} 
	\end{bmatrix},
\end{equation}

\begin{equation}
	I_2^2 = \begin{bmatrix}
		I_{x_2} & 0 &0 \\
		0 & I_{y_2} &0 \\
		0 & 0 & I_{z_2} 
	\end{bmatrix}
\end{equation}

a jakobiány translační

\begin{equation}
	J_{1, \text{TR}} = \begin{bmatrix}
		-\frac{l_1}{2} \sin(\theta_1) & 0 \\
		\frac{l_1}{2} \cos(\theta_1) & 0 \\
		0 & 0
	\end{bmatrix},
\end{equation}

\begin{equation}
	J_{2, \text{TR}} = \begin{bmatrix}
		-l_1 \sin(\theta_1) - \frac{l_2}{2} \sin(\theta_1) \cos(\theta_2) & -\frac{l_2}{2} \cos(\theta_1) \sin(\theta_2) \\
		l_1 \cos(\theta_1) + \frac{l_2}{2} \cos(\theta_1) \cos(\theta_2) & -\frac{l_2}{2} \sin(\theta_1) \sin(\theta_2) \\
		0 & - \frac{l_2}{2} \cos(\theta_2)
	\end{bmatrix},
\end{equation}

i rotační

\begin{equation}
	J_{1, \text{ROT}} = \begin{bmatrix}
		0 & 0 \\ 
		0 & 0 \\ 
		1 & 0
	\end{bmatrix},
\end{equation}

\begin{equation}
	J_{2, \text{ROT}} = \begin{bmatrix}
		0 & -\sin(\theta_1) \\
		0 & \cos(\theta_1) \\
		1 & 0
	\end{bmatrix}.
\end{equation}

\section{Zobecnělá matice setrvačnosti}

Z definice platí

\begin{equation}
	M(q) = \sum_{i=1}^2 \left( J_{i, \text{TR}}^\text{T} \cdot m_i \cdot J_{i, \text{TR}} + J_{i, \text{ROT}}^\text{T}
	\cdot R_i^0 \cdot I_i^i \cdot (R_i^0)^\text{T} \cdot J_{i, \text{ROT}} \right).
\end{equation}

Po dosazení vyjde matice
\begin{equation}
	\mathbf{M}(q) = \begin{bmatrix}
		M_{11} & M_{12} \\
		M_{21} & M_{22}
	\end{bmatrix},
	\label{eq:M}
\end{equation}
kde dílčí prvky $M_{ij}$ jsou
\begin{equation}
	\begin{split}
		M_{11} &= \mathrm{Ix}_{2} \sin^2(\theta_2) +\mathrm{Iy}_{1}+\frac{{l_{1}}^2\,m_{1}}{4}+{l_{1}}^2\,m_{2}+\mathrm{Iy}_{2}\,{\cos\left(\theta _{2}\right)}^2+\frac{{l_{2}}^2\,m_{2}\,{\cos\left(\theta _{2}\right)}^2}{4}+l_{1}\,l_{2}\,m_{2}\,\cos\left(\theta _{2}\right), \\
		M_{12} &= 0, \\
		M_{21} &= 0, \\
		M_{22} &= \frac{m_{2}\,{l_{2}}^2}{4}+\mathrm{Iz}_{2}.
	\end{split}
\end{equation}

\section{Potenciální energie}

Z definice platí
\begin{equation}
	\mathcal{V} = \sum_{i=1}^2 (m_i \cdot \begin{bmatrix}
		-g & 0 & 0
	\end{bmatrix} \cdot r^0_{c_i}),
\end{equation}
tedy dílčí potenciální energie lze získat jako součin hmotnosti $i$-tého ramene a průměru polohy jeho těžiště to svislé osy.
Vzhledem k orientaci světového souřadného systému je svislý směr reprezentován osou $x$ a tedy první složkou vektoru.

Po dosazení konkrétních proměnných
\begin{equation}
	\mathcal{V} = -g\,\left(m_{2}\,\left(l_{1}\,\cos\left(\theta _{1}\right)+\frac{l_{2}\,\cos\left(\theta _{1}\right)\,\cos\left(\theta _{2}\right)}{2}\right)+\frac{l_{1}\,m_{1}\,\cos\left(\theta _{1}\right)}{2}\right)
\end{equation}
a po parciálním derivování
\begin{equation}
	\mathbf{G}(q) = \begin{bmatrix}
		g\,\left(m_{2}\,\left(l_{1}\,\sin\left(\theta _{1}\right)+\frac{l_{2}\,\cos\left(\theta _{2}\right)\,\sin\left(\theta _{1}\right)}{2}\right)+\frac{l_{1}\,m_{1}\,\sin\left(\theta _{1}\right)}{2}\right) \\
		\frac{g\,l_{2}\,m_{2}\,\cos\left(\theta _{1}\right)\,\sin\left(\theta _{2}\right)}{2}
	\end{bmatrix}.
	\label{eq:G}
\end{equation}

\section{Coriolisova matice}

Z definice Christoffelových symbolů platí
\begin{equation}
	\begin{split}
		c_{ijk}(q) &= \frac{1}{2} \left(\frac{\partial M_{ki}}{\partial q_j} + \frac{\partial M_{kj}}{\partial q_i} + \frac{\partial M_{ij}}{\partial q_k}\right), \\
		C_{kj}(q, \dot{q}) &= \sum_{i=1}^2 c_{ijk}(q) \cdot \dot{q_i},
	\end{split}
\end{equation}
Coriolisova matice má poté tvar
\begin{equation}
	\mathbf{C}(q, \dot{q}) = \begin{bmatrix}
		C_{11}(q, \dot{q}) & C_{12}(q, \dot{q}) \\
		C_{21}(q, \dot{q}) & C_{22}(q, \dot{q})
	\end{bmatrix},
	\label{eq:C}
\end{equation}

kde dílčí prvky jsou
\begin{equation}
	\begin{split}
		C_{11}(q, \dot{q}) &= -\dot{\theta}_{2}\,\left(\frac{m_{2}\,\sin\left(2\,\theta _{2}\right)\,{l_{2}}^2}{8}+\frac{l_{1}\,m_{2}\,\sin\left(\theta _{2}\right)\,l_{2}}{2}-\frac{\mathrm{Ix}_{2}\,\sin\left(2\,\theta _{2}\right)}{2}+\frac{\mathrm{Iy}_{2}\,\sin\left(2\,\theta _{2}\right)}{2}\right) ,\\
		C_{12}(q, \dot{q}) &= -\frac{\dot{\theta}_{1}\,\left(\frac{m_{2}\,\sin\left(2\,\theta _{2}\right)\,{l_{2}}^2}{2}+2\,l_{1}\,m_{2}\,\sin\left(\theta _{2}\right)\,l_{2}-2\,\mathrm{Ix}_{2}\,\sin\left(2\,\theta _{2}\right)+2\,\mathrm{Iy}_{2}\,\sin\left(2\,\theta _{2}\right)\right)}{4} ,\\
		C_{21}(q, \dot{q}) &= -\frac{\dot{\theta}_{1}\,\left(m_{2}\,\sin\left(2\,\theta _{2}\right)\,{l_{2}}^2+4\,l_{1}\,m_{2}\,\sin\left(\theta _{2}\right)\,l_{2}-4\,\mathrm{Ix}_{2}\,\sin\left(2\,\theta _{2}\right)+4\,\mathrm{Iy}_{2}\,\sin\left(2\,\theta _{2}\right)\right)}{8} ,\\
		C_{22}(q, \dot{q}) &= 0.
	\end{split}
\end{equation}

\section{Lagrangeova rovnice}

S použitím matic vypočtených v rovnicích \eqref{eq:G}, \eqref{eq:M} a \eqref{eq:C} lze Lagrangeovu rovnici studovaného systému zapsat ve tvaru
\begin{equation}
	\mathbf{M}(q) \ddot{q} + \mathbf{C}(q, \dot{q}) \dot{q} + \mathbf{G}(q) = Q,
\end{equation}
kde $Q$ reprezentuje dále neupřesněné vstupní síly a momenty.


\end{document}


